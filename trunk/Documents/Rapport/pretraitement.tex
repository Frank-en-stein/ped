Prétraitement : 
Nos images sont bruités  


Filtre passe bas, flou gaussien ...   
et 
augmentation de contraste...

sont des outils que nous avons utilisé pendant le projet   


étude du filtre de Weiner qui permet de flouter les zones homogéne et de réhausser les contrastes 


Avant de commencer la détection de nouveautés ou de forme nous devons segmenter notre image afin de détecter les objets sur l'image.

Etant donné le type de support, un document avec des symboles noirs sur fond blanc, nous avons focaliser notre travail sur la binarisation.

La binarisation à pour but de ségmenter l'image en 2 classes, le fond et l'image.
Il existe différentes techniques pour binariser une image.

Nous avons dans un premier temps étudié la binarisation par seuillage global, on choisit la classe du pixel à partir d'un seuil.

\begin{equation}
	\forall i,j \in \mathbf{M*N} \quad I(i,j)=
	\left\lbrace
	\begin{array}{ccc}
		1 &\mbox{si}& f(i,j) > S,\\
		0 &\mbox{sinon}&
	\end{array}\right.
\end{equation}

%%Avec : 
%%\begin{description}
%%  \item[ $N × M$] nombre de colonnes et de lignes de l’image ;
%%  \item[$I$] image binarisée ;
%%  \item[$S$] seuil de binarisation.
%%  \item[$f$] valeur fonction de l’image d’origine ;\ldots
%%  
%\end{description}

Le probléme est que l'on utilise des documents dégradés ( mauvaise ilumination et papier marqué) donc bien souvent il n'existe pas de seuil global.
'''''''Cf image'''''''''' 


Ensuite nous avons étudié la detection de contour car un dessin est une accumulation de trait.
Nous avons donc appliqué le filtre de Sobel et Canny à nos images.

Filtre de Sobel

Filtre de canny 



Les résultats étaient bien sur èfficace sur les traits fin mais moins bon dans les zones coloriées.
Toutefois il est possible d'obtenir, en y combinant des filtres morphologiques, les zones dans lesquelles il y a un objet.



L'étude de ,qui compare les différents algorithmes de binarisation emet un classement des meilleurs algorithmes pour les images dégradés.
La methode de kittler se basant sur le clustering ... nous avons abandonné cette méthode car les résultats obtenus étaient insuffisants.

et les méthodes par seuillage local, son principe est d’utiliser une étude localisée autour du pixel pour déterminer quel seuil utiliser. Pour réaliser cette étude locale, les techniques utilisent une fenêtre d’étude centrée sur le pixel à étudier. Cette fenêtre peut avoir différentes tailles, de nombreuses méthodes on été proposé bernsten, Niblack , Sauvola.

Sauvola est indiqué comme obtenant les meilleurs résultats

\begin{equation}
	S(i,j) = \mu(i,j) + \kappa.((\sigma(i,j)/R)-1))
\end{equation}

Avec :
- S(i, j) : seuil à appliquer pour le point i, j ;\\
- $\sigma(i, j)$ : valeur de l’écart type dans une fenêtre centré en i, j de taille $N * M$ ;\\
- $\mu(i, j)$ : valeur moyenne des niveaux de gris dans la même fenêtre ;\\
- $\kappa$ : constante fixée le plus généralement à 0, 2 ;\\
- R : constante permettant d'ajuster la dynamique de l'ecart type, géneralement 128.
- N et M appartenant à N.\\

La méthode de Sauvola calcule le seuil de binarisation en fonction de la moyenne et de l'ectart type des pixels ajuster par la constante R , le raport entre les deux est défini par la constante $\kappa$.
Sauvola conseille un $\kappa \in \left[ 0.2 ;0.5 \right[$, mais ce type de paramétre est fixé pour binariser des textes, or nos documents sont des dessins effectué au crayons a papier et peuvent avoir des trait trés fins.
dans son article  , fixe $\kappa$ à 0.05, ce qui augmente l'influence de l'ecart type et permet d'obtenir de meilleurs résultats, notament lorsqu'il y a un faible contraste comme dans notre cas.   

La taille de la fenêtre est fixé en fonction de la taille de l'image, nous ne prenons pas la même taille de fenetre pour l'image en haute résolution que pour le flux vidéo en basse résolution.

Optimisation.

Les images integrales sont une représentation sous la forme d'une image, de même taille que l'image d'origine, qui en chacun de ses points contient la somme des pixels situés au dessus et à gauche de ce point. Plus formellement, l'image intégrale ii est définie à partir de l'image i par : 

\begin{equation}
	%ii(x,y) = \sum \liminf{\underset{x' \leq x}{y' \leq y }} i(x',y')
	ii(x,y) = \sum_{x' \leq x  \atop y' \leq y } i(x',y')
\end{equation}







résumé des differentes techniques 


Seuillage global

detection de contour + filtre morphologique 
     
Seuillage local Opencv moyenne et gaussian

seuillage locale Sauvola
