\subsection{Presentation des logiciels de test}

\subsubsection{Reconnaissance de formes simples dans un dessin}

Le premier logiciel que nous avons développé est un logiciel de reconnaissance de formes simples dans un dessin. Ce logiciel commence par lire une vidéo. L'utilisateur peut à tout moment mettre en pause la vidéo, et prendre des captures de la vidéo.

Lorsqu'une capture est prise, notre logiciel effectue des pré-traitements sur l'image obtenue, puis lance l'analyse de template matching. Le logiciel affiche ensuite le résultat de l'analyse, à savoir la capture prise dans la vidéo enrichie de \texttt{Bounding Box} autour des objets détectés.

Il est ensuite possible de naviguer entre les différentes captures.

\subsubsection{Détection des évolutions d'un dessin}

Le second logiciel développé est un logiciel de détection des évolutions d'un dessin.
Ce logiciel propose de comparer deux images d'un dessin. Il affiche ensuite les différences entre ces deux images.